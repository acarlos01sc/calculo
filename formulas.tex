\documentclass[
    12pt, % Font size
    openright,
    twoside, % Use both sides of the paper
    a4paper, % Paper size
    article,
    english,brazil % languages: the last one is the default of the text
]{abntex2}

\usepackage{geometry}
 \geometry{
 a4paper,
 total={170mm,257mm},
 left=20mm,
 top=20mm
 }

% Define font 
\usepackage{fourier}
\usepackage{amsmath}

% Table type
\usepackage{tabularx}

% Graphics
\usepackage{graphicx}

% accentuation support according to the compiler used
\usepackage{ifxetex}
\ifxetex
    \usepackage{fontspec}
    \defaultfontfeatures{Ligatures={TeX}}
\else
    \usepackage[utf8]{inputenc}
    \usepackage[T1]{fontenc}
\fi
% end of accentuation configuration

%\usepackage{indentfirst}
\usepackage{color}
\setlength{\parskip}{0.5em}

\author{Antonio Carlos Soares Cardoso}
\title{Física - Mecânica Clássica}
\local{Brasília, DF}
\data{Maio - 2022}

%Formating section
\usepackage{titlesec}
\titleformat{\section}[hang]
{\normalfont\bfseries\filright}
{\Large\thesection.}{1ex}{\Large}
[\vspace{1ex}%
{\titlerule[2pt]}]

%Plots
\usepackage{pgfplots}

%Formating subsection
\titleformat{\subsection}[hang]
{\normalfont\bfseries\filright}
{\thesubsection}{1ex}{}

%Formating Table of Contents
\makeatletter
\renewcommand\tableofcontents{%
  \null\hfill\textbf{\Large\contentsname}\hfill\null\par
  \@mkboth{\MakeUppercase\contentsname}{\MakeUppercase\contentsname}%
  \@starttoc{toc}%
}
\makeatother

\begin{document}

\tableofcontents

\newpage

\textual

\noindent

\chapter{Mecânica Clássica} 

\section{Cinemática}

$V = V_o + at$ \\
$S = S_o + V_ot + \frac{at^2}{2}$ \\
$V^2 = V_o^2 + 2a\Delta S$ \\

Estas fórmulas são válidas quando a aceleração $(a)$ for constante.

\section{Dinâmica}

$q = mv$ \quad \textbf{Quantidade de Movimento.} \\
$F = \frac{dp}{dt}$ \\
$F = ma$ \quad \textbf{Força - Unidade newton $(N)$.} \\
$I = F\Delta t$ \quad \textbf{Impulso.} \\
$\tau = Frsin\theta$ \quad \textbf{Torque.} \\

\subsection{Tipos de Força} 

$P = mg$ \\
$F_{el} = kx$ \\
$F_{at} = \mu N$ \\
$F_{g} = G\frac{M_1M_2}{d^2}$ onde $G = 6,67 \cdot 10^{-11} \frac{Nm^2}{kg^2}$

\subsection{Movimento Circular}

$a_c = \frac{v^2}{r}$ \\
$F_c = \frac{mv^2}{r}$ \\

\subsection{Trabalho e Energia}

$T = Fdcos\theta = \Delta E$ \quad \textbf{Trabalho - Unidade joule $(J)$.} \\
$E_c = \frac{mv^2}{2}$ \\
$E_{el} = \frac{kx^2}{2}$ \\
$E_{p} = mgh$ \\

\subsection{Potência}

$P = \frac{T}{\Delta t}$ \quad \textbf{Potência - Unidade watt ou $J/s$ $(W)$.} \\
$P = F \cdot v$ \\

\chapter{Eletromagnetismo}

\section{Eletricidade}

\subsection{Eletrostática}

$F_{elet} = k\frac{q_1q_2}{r^2}$ onde $k$ no vácuo vale $9 \cdot 10^9 \frac{Nm^2}{C^2}$ \\
$E = k\frac{Q}{d^2}$ \\
$F_{elet} = Eq$ \\

\subsection{Eletrodinâmica}

$i = \frac{\Delta Q}{\Delta t}$ \quad \textbf{Corrente Elétrica - Unidade ampère $(A)$.}\\

\end{document}