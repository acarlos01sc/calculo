\documentclass[
    12pt, % Font size
    openright,
    twoside, % Use both sides of the paper
    a4paper, % Paper size
    article,
    english,brazil % languages: the last one is the default of the text
]{abntex2}

\usepackage{geometry}
 \geometry{
 a4paper,
 total={170mm,257mm},
 left=20mm,
 top=20mm
 }

% Define font 
\usepackage{fourier}
\usepackage{amsmath}

% accentuation support according to the compiler used
\usepackage{ifxetex}
\ifxetex
    \usepackage{fontspec}
    \defaultfontfeatures{Ligatures={TeX}}
\else
    \usepackage[utf8]{inputenc}
    \usepackage[T1]{fontenc}
\fi
% end of accentuation configuration

\usepackage{indentfirst}
\usepackage{color}
\setlength{\parskip}{0.5em}

\author{Antonio Carlos Soares Cardoso}
\title{Cálculo Diferencial e Integral}
\local{Brasília, DF}
\data{Março - 2021}

%Formating section
\usepackage{titlesec}
\titleformat{\section}[hang]
{\normalfont\bfseries\filright}
{\Large\thesection.}{1ex}{\Large}
[\vspace{1ex}%
{\titlerule[2pt]}]

%Formating subsection
\titleformat{\subsection}[hang]
{\normalfont\bfseries\filright}
{\thesubsection}{1ex}{}

%Formating Table of Contents
\makeatletter
\renewcommand\tableofcontents{%
  \null\hfill\textbf{\Large\contentsname}\hfill\null\par
  \@mkboth{\MakeUppercase\contentsname}{\MakeUppercase\contentsname}%
  \@starttoc{toc}%
}
\makeatother

\begin{document}

\begin{capa}
    \center
    \ABNTEXchapterfont\Large Cálculo\\
    \vspace*{1cm}
    {\ABNTEXchapterfont\large\imprimirautor}
    \vfill
    \begin{center}
    \ABNTEXchapterfont\bfseries\LARGE\imprimirtitulo
    \end{center}
    \vfill
    \large\imprimirlocal \\
    \large\imprimirdata
    \vspace*{1cm}
\end{capa}

\tableofcontents

\newpage

\textual

\section{Propriedades Básicas dos Números}

Cálculo, Álgebra e Aritmética são ramos da Matemática que definem operações realizadas com números. Desta forma, é necessáro definir-se as propriedades básicas dos números, sobre as quais as demais construções do conhecimento numérico são erigidas.

Neste capítulo serão apresentadas as 12 propriedades básicas dos números.

\subsection{Propriedade 1 - Associatividade da Adição.}

Se $a,b$ e $c$ são números quaisquer, então: 
$$ a+(b+c)=(a+b)+c $$

Esta afirmativa parece óbvia, mas ela é necessária para a prova de outras propriedades matemáticas não tão óbvias. O cerne desta sentença é que, para quaisquer números, a ordem dos parênteses é irrelevante na adição. 

\subsection{Propriedade 2 - Identidade na Adição}
Se $a$ é um número qualquer, então: 

$$a+0=0+a=a$$

Esta sentença afirma que o número $0$ não altera o resultado nas operações de adição.

\subsection{Propriedade 3 - Adição de Inversos}
Para qualquer número $a$, existe um número $-a$ tal que: 

$$a+(-a)=(-a)+a=0$$

As três propriedades até aqui listadas nos permite, de maneira formal, fazer o cálculo de equações com incógnitas e que envolvam apenas adição. 

Exemplo: Seja $a$ um número qualquer, calcule o valor de $x$ sabendo-se que $x+a=a$. 

$$x+a=a$$
$$x+a+(-a)=a+(-a)$$
$$x+0=0$$
$$x=0$$

\subsection{Propriedade 4 - Comutatividade na Adição}
Se $a$ e $b$ são números quaisquer, então: 

$$a+b=b+a$$

Esta e a Propriedade 1 asseguram que operações de adição não dependem da ordem em que são executadas.

\subsection{Propriedade 5 - Associatividade na Multiplicação}
Se $a,b$ e $c$ são números quaisquer, então: 
$$ a \cdot (b \cdot c)=(a \cdot b)\cdot c $$

Também para a multiplicação, a ordem dos parênteses não altera o resultado quando a sentença contém apenas a multiplicação.

\subsection{Propriedade 6 - Identidade na Multiplicação}
Se $a$ é um número qualquer, então: 

$$a \cdot 1=1 \cdot a=a$$

Além disso, $1 \neq 0$

Na multiplicação o número $1$ não altera o resultado.

A assertiva de que $1 \neq 0$ parece desnecessária, porém, como não definimos até este ponto o resultado de $a \cdot 0$ ela torna-se necessária para evitar que se obtenha um resultado absurdo.

\subsection{Propriedade 7 - Multiplicação de Inversos}
Se $a$ é um número qualquer diferente de $0$, então: 

$$a \cdot a^{-1}=1$$

Note-se a restrição da propriedade de que esta é válida apenas quando $a \neq 0$.

\subsection{Propriedade 8 - Comutatividade da Multiplicação}
Se $a$ e $b$ são números quaisquer, então: 

$$a \cdot b=b \cdot a$$

\subsection{Propriedade 9 - Distributividade da Multiplicação}
Se $a$, $b$ e $c$ são números quaisquer, então: 

$$a \cdot (b+c)=a \cdot b+a \cdot c$$

Com a propriedade 9 é possível demonstrar que $a \cdot 0 = 0$. 

Seja $c = 0$ e $a, b$ números quaiquer, então: 

$$a \cdot (b+c)=a \cdot b+a \cdot c$$
$$a \cdot (b+0)=a \cdot b+a \cdot 0$$
$$a \cdot b=a \cdot b+a \cdot 0$$
$$-(a \cdot b)+a \cdot b=-(a \cdot b)+a \cdot b+a \cdot 0$$
$$0=0+a \cdot 0$$
$$0=a \cdot 0$$ 

Com as propriedades de 1 a 9 em mãos, é possível provar-se porque na multiplicação entre dois números negativos o resultado será positivo. 

Sejam $a,b$ números positivos quaisquer, então:

$$(-a) \cdot (-b)+(-a) \cdot b=(-a) \cdot (-b+b)$$
$$(-a) \cdot (-b)+(-a) \cdot b=(-a) \cdot (0)$$
$$(-a) \cdot (-b)+(-a) \cdot b=0$$
$$(-a) \cdot (-b)+(-a) \cdot b+a \cdot b=a \cdot b$$
$$(-a) \cdot (-b)+0=a \cdot b$$
$$(-a) \cdot (-b)=a \cdot b$$
\end{document}