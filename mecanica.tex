\documentclass[
    12pt, % Font size
    openright,
    twoside, % Use both sides of the paper
    a4paper, % Paper size
    article,
    english,brazil % languages: the last one is the default of the text
]{abntex2}

\usepackage{geometry}
 \geometry{
 a4paper,
 total={170mm,257mm},
 left=20mm,
 top=20mm
 }

% Define font 
\usepackage{fourier}
\usepackage{amsmath}

% accentuation support according to the compiler used
\usepackage{ifxetex}
\ifxetex
    \usepackage{fontspec}
    \defaultfontfeatures{Ligatures={TeX}}
\else
    \usepackage[utf8]{inputenc}
    \usepackage[T1]{fontenc}
\fi
% end of accentuation configuration

\usepackage{indentfirst}
\usepackage{color}
\setlength{\parskip}{0.5em}

\author{Antonio Carlos Soares Cardoso}
\title{Cálculo Diferencial e Integral}
\local{Brasília, DF}
\data{Março - 2021}

%Formating section
\usepackage{titlesec}
\titleformat{\section}[hang]
{\normalfont\bfseries\filright}
{\Large\thesection.}{1ex}{\Large}
[\vspace{1ex}%
{\titlerule[2pt]}]

%Formating subsection
\titleformat{\subsection}[hang]
{\normalfont\bfseries\filright}
{\thesubsection}{1ex}{}

%Formating Table of Contents
\makeatletter
\renewcommand\tableofcontents{%
  \null\hfill\textbf{\Large\contentsname}\hfill\null\par
  \@mkboth{\MakeUppercase\contentsname}{\MakeUppercase\contentsname}%
  \@starttoc{toc}%
}
\makeatother

\begin{document}

\tableofcontents

\newpage

\textual

\section{Mecânica Clássica}

A Mecânica Clássica corresponde à área da Física que estuda o movimento dos corpos macroscópicos. Para fins didáticos pode ser subdividida em:

\item \textbf{Cinemática} - Estuda o Movimento dos corpos em si, sem deter-se na análise de suas causas.
\item \textbf{Dinâmica} - Analisa as causas que podem provocar movimento nos corpos.
\item \textbf{Estática} - Analisa os corpos sem movimento aparente.

\section{Cinemática}

Na cinemática busca-se conhecer o comportamento da posição $(s)$ de um corpo em movimento em função do tempo. Ao analisar a posição do corpo em função do tempo teremos uma curva, cuja taxa de variação ou derivada corresponderá à velocidade $(v)$ deste corpo. Assim:

\textbf{Velocidade} - Corresponde à variação do espaço percorrido por um corpo em função do tempo, ou seja:

$$ v = \frac{ds}{dt} $$

Quando a velocidade do corpo em análise também varia com o tempo, esta variação corresponderá à aceleração.

\textbf{Aceleração} - Corresponde à taxa de variação temporal da velocidade de um corpo, ou seja:

$$ a = \frac{dv}{dt} = \frac{d^2s}{dt^2}$$

\subsection{Aceleração constante}

Em conformidade com a definição de aceleração e para o caso específico desta ser uma constante $(a)$, obtém-se as seguintes equações descritivas do movimento:

$$ a = cte. $$
$$ \int \frac{dv}{dt} dt = \int a dt$$
$$ v(t) = v_0 + at $$

Então:

$$ \int \frac{ds}{dt} dt = \int v dt = \int (v_0 + at) dt $$
$$ s(t) = s_0 + v_0 t + \frac{a t^2}{2} $$

Expressando-se $t$ em função de $v$ e substituindo em $s(t)$:

$$t = \frac{(v - v_0)}{a}$$
$$s = s_0 + \frac{(v - v_0)}{a}(v_0 + \frac{a}{2}\frac{(v - v_0)}{a})$$

Após algumas manipulações algébricas obtém-se a equação de Torricelli. 

$$v^2 = v_0^2 + 2a(s-s_0)$$

Resumidamente, as equações para análise do movimento quando a aceleração é constante (inclusive para o caso desta ser zero) são:

\begin{equation}
v(t) = v_0 + at
\end{equation}
\begin{equation}
s(t) = s_0 + v_0t + \frac{at^2}{2}    
\end{equation}
\begin{equation}
v^2 = v_0^2 + 2a(s - s_0)
\end{equation}


\end{document}